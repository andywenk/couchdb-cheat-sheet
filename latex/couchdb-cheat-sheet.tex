\documentclass[19pt,landscape,twocolumn]{article}

% fonts
\usepackage[utf8]{inputenc}
\usepackage[T1]{fontenc}
\usepackage{times}
\usepackage{pifont} % use it with \ding{number} | http://en.wikibooks.org/wiki/LaTeX/Formatting

% document format
\usepackage{lscape}
\usepackage[top=1cm, bottom=1.5cm, left=1cm, right=1cm]{geometry}

% footer
\usepackage{fancyhdr}
\pagestyle{fancyplain}
\fancyhf{}
\renewcommand{\headrulewidth}{0pt}
\renewcommand{\footrulewidth}{0pt}
\lfoot{\fancyplain{}{CouchDB Cheat Sheet - Andreas Wenk}}
\rfoot{\fancyplain{}{\thepage}}

% section formatting
\usepackage{titlesec}
\titleformat{\section}{\bf\fontsize{12}{12}\selectfont}{\thesection\quad}{0em}{}
\titleformat{\subsection}{\bf\fontsize{10}{10}\selectfont}{\thesubsection\quad}{0em}{}

% defined commands
\newcommand{\mono}[1]{\texttt{\textendash\textendash {#1}}}
\newcommand{\htmlverb}[1]{{[}\textbf{{#1}}{]}}

\begin{document}
% set the font size for the whole document
\fontsize{9}{10}\selectfont

% title
\title{CouchDB-Cheat-Sheet}
\author{Andreas Wenk -- Version: 0.0.1 -- August 2011}
\date{}
\maketitle

\begin{abstract}
Dieses Cheat-Sheet ist eine Übersicht der RESTful CouchDB-API.
\end{abstract}

\section{Überblick}

\section{Datenbanken}
\subsection{Datenbankinformationen erhalten \htmlverb{GET}}
\begin{quote}
\begin{verbatim}
curl -X GET http://127.0.0.1:5984/datenbankname
{"db_name":"datenbankname",
 "doc_count":21,
 "doc_del_count":56,
 "update_seq":454,
 "purge_seq":0,
 "compact_running":false,
 "disk_size":454750,
 "instance_start_time":"1311337314684694",
 "disk_format_version":5,
 "committed_update_seq":454}
\end{verbatim}
\end{quote}

\subsection{Alle Datenbanken im Cluster \htmlverb{GET}}
\begin{quote}
\begin{verbatim}
curl -X GET http://127.0.0.1:5984/_all_dbs
["_replicator","_users"]
\end{verbatim}
\end{quote}

\subsection{Datenbank erstellen \htmlverb{PUT}}
\begin{quote}
\begin{verbatim}
curl -X PUT http://127.0.0.1:5984/datenbankname
{"ok": true}
\end{verbatim}
\end{quote}

\subsection{Datenbank löschen \htmlverb{DELETE}}
\begin{quote}
\begin{verbatim}
curl -X DELETE http://127.0.0.1:5984/datenbankname
{"ok": true}
\end{verbatim}
\end{quote}

\subsection{Datenbank replizieren \htmlverb{POST}}
\begin{quote}
\begin{verbatim}
curl -X POST http://127.0.0.1:5984/_replicate 
     -H "content-type:application/json" 
     -d '{"source": "datenbankname", 
          "target": "datenbankname_neu"}'
{"ok":true, "session_id":"f75eb944bac70f40e77953f484afb64c", 
 "source_last_seq":36, "history":
   [{"session_id":"f75eb944bac70f40e77953f484afb64c",
     "start_time":"Thu, 14 Apr 2011 20:36:12 GMT", 
     "end_time":"Thu, 14 Apr 2011 20:36:12 GMT",
     "start_last_seq":0,
     "end_last_seq":36,
     "recorded_seq":36,
     "missing_checked":0,
     "missing_found":14,
     "docs_read":14,
     "docs_written":14,
     "doc_write_failures":0
  }]
}
\end{verbatim}
\end{quote}

\section{Dokumente}
\subsection{UUIDs ausgeben \htmlverb{GET}}
\emph{(params)} count={[}n{]}
\begin{quote}
\begin{verbatim}
curl -X GET http://127.0.0.1:5984/_uuids?count=3
{"uuids":
  ["b7fc0ca7dffcc6d0f240c1e5bb000998",
   "b7fc0ca7dffcc6d0f240c1e5bb000fe8",
   "b7fc0ca7dffcc6d0f240c1e5bb001c9a"]
}
\end{verbatim}
\end{quote}

\subsection{Alle Dokumente erhalten \htmlverb{GET}}
\emph{(params)} descending=true

\subsection{Dokument erstellen \htmlverb{PUT}}
\begin{quote}
\begin{verbatim}
curl -X PUT http://127.0.0.1:5984/datenbankname/ \
  b7fc0ca7dffcc6d0f240c1e5bb000998
{"ok":true,"id":"b7fc0ca7dffcc6d0f240c1e5bb000998",
 "rev":"1-967a00dff5e02add41819138abb3284d"}
\end{verbatim}
\end{quote}

\subsection{Dokument erstellen \htmlverb{POST}}
\begin{quote}
\begin{verbatim}
curl -X POST http://127.0.0.1:5984/datenbankname/
     -H "content-type: application/json"
     -d '{}'
{"ok":true,"id":"b7fc0ca7dffcc6d0f240c1e5bb000fe8",
 "rev":"1-367b00dfc5e02axd41819138abb3284d"}
\end{verbatim}
\end{quote}

\subsection{Dokument anfragen \htmlverb{GET}}
\emph{(params)} rev={[}revision{]}, revs=true, revs\_info=true
\begin{quote}
\begin{verbatim}
curl -X GET http://127.0.0.1:5984/datenbankname/ \
  b7fc0ca7dffcc6d0f240c1e5bb000998
{"_id":"b7fc0ca7dffcc6d0f240c1e5bb000998",
 "_rev":"1-367b00dfc5e02axd41819138abb3284d",
 "inhalt":"hier steht was"}
\end{verbatim}
\end{quote}

\subsection{Dokument erweitern / aktualisieren \htmlverb{PUT}}
\begin{quote}
\begin{verbatim}
curl -X PUT http://127.0.0.1:5984/datenbankname/ \
  b7fc0ca7dffcc6d0f240c1e5bb000998
     -d '{"_rev":"1-367b00dfc5e02axd41819138abb3284d",
          "neuer Inhalt": "kommt hier hin",
          "inhalt": "bestehender wird aktualisiert"}'
{"ok":true,"id":"7fc0ca7dffcc6d0f240c1e5bb000998",
 "rev":"2-53f21467344e4cb88384fc9e2e189049"}
\end{verbatim}
\end{quote}

\subsection{Dokument-Attachment speichern \htmlverb{PUT}}
\emph{(!)} generell könnte auch nur die Option \mono{data} für eine reine Textdatei verwendet werden
\begin{quote}
\begin{verbatim}
curl -X PUT http://127.0.0.1:5984/datenbankname/ \
  b7fc0ca7dffcc6d0f240c1e5bb000998/ \
  geburtstag.txt?rev=2-53f21467344e4cb88384fc9e2e189049
     --data-binary @geburtstag.txt
     -H "content-type: text/plain;charset=utf-8"
HTTP/1.1 201 Created
Server: CouchDB/1.1.0 (Erlang OTP/R14B03)
Location: http://127.0.0.1:5984/datenbankname/ \
  b7fc0ca7dffcc6d0f240c1e5bb000998/geburtstag.txt
Etag: "3-d5171a67e9e27aeba9beac32149a86a9"
Date: Mon, 07 Feb 2011 22:53:25 GMT
Content-Type: text/plain;charset=utf-8
Content-Length: 95
Cache-Control: must-revalidate

{"ok":true,"id":"b7fc0ca7dffcc6d0f240c1e5bb000998",
 "rev":"3-d5171a67e9e27aeba9beac32149a86a9"}
\end{verbatim}
\end{quote}

\subsection{Dokument löschen \htmlverb{DELETE}}
\begin{quote}
\begin{verbatim}
curl -X DELETE http://127.0.0.1:5984/datenbankname/ \
  b7fc0ca7dffcc6d0f240c1e5bb000998 \
  ?rev=3-d5171a67e9e27aeba9beac32149a86a9
{"ok":true,"id":"b7fc0ca7dffcc6d0f240c1e5bb000998",
 "rev":"3-d5171a67e9e27aeba9beac32149a86a9"}
\end{verbatim}
\end{quote}

\subsection{Dokument Info \htmlverb{HEAD}}

\subsection{Dokument kopieren \htmlverb{COPY}}

\subsection{Mehrere Dokumente gleichzeitig anlegen \htmlverb{POST}}

\end{document}
