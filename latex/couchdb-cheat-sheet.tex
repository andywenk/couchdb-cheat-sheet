\documentclass[19pt,landscape,twocolumn]{article}

% PDF Bauen: pdflatex ./couchdb-cheat-sheet.tex

% fonts
\usepackage[utf8]{inputenc}
\usepackage[T1]{fontenc}
\usepackage[german]{babel}

% document format
\usepackage{lscape}
\usepackage[top=1cm, bottom=1.5cm, left=1cm, right=1cm]{geometry}
\setlength{\parskip}{1.65mm plus0mm minus1mm}

% listings
\usepackage{fancyvrb} % http://ftp.math.purdue.edu/mirrors/ctan.org/macros/latex/contrib/fancyvrb/fancyvrb.pdf
\usepackage{color} % http://en.wikibooks.org/wiki/LaTeX/Colors
\definecolor{midnightblue}{rgb}{0.098,0.098,0.439}

% footer
\usepackage{fancyhdr} % ftp://ftp.mpi-sb.mpg.de/pub/tex/mirror/ftp.dante.de/pub/tex/macros/latex/contrib/fancyhdr/fancyhdr.pdf
\pagestyle{fancyplain}
\fancyhf{}
\renewcommand{\headrulewidth}{0pt}
\renewcommand{\footrulewidth}{0pt}
\lfoot{\fancyplain{}{CouchDB Cheat Sheet - Andreas Wenk}}
\rfoot{\fancyplain{}{\thepage}}

% URLs
\usepackage[pdfborder={0 0 0},urlcolor=blue,colorlinks=true,citecolor=blue]{hyperref}

% section formatting
\usepackage{titlesec}
\titleformat{\section}{\bf\fontsize{12}{12}\selectfont}{\thesection\quad}{0em}{}
\titleformat{\subsection}{\bf\fontsize{10}{10}\selectfont}{\thesubsection\quad}{0em}{}

% my commands
\newcommand{\mono}[1]{\texttt{\textendash\textendash {#1}}}
\newcommand{\htmlverb}[1]{{[}\textbf{{#1}}{]}}
\newcommand{\brackets}[1]{{[}{#1}{]}}

\DefineVerbatimEnvironment{code}{Verbatim}{fontsize=\small,xleftmargin=5mm,formatcom=\color{midnightblue}}

\begin{document}
% set the font size for the whole document
\fontsize{9}{10}\selectfont

% title
\title{CouchDB-Cheat-Sheet}
\author{Andreas Wenk -- Version: 0.1.0 -- August 2011}
\date{}
\maketitle

\begin{abstract}
Dieses Cheat-Sheet ist eine Übersicht der RESTful API von CouchDB \cite{1}. Das Ziel ist es,
die gebräuchlichsten Requests jeweils anhand eines Beispiels zu zeigen. Unter \cite{2}
ist eine CouchDB zu finden, in der die Beispiele nachgestellt werden können (Basic-Auth: ohne 
Daten `ok' klicken). 

Eine recht gute API Referenz ist unter \cite{3} zu finden.

Damit die Listings so kurz als möglich sind, heisst die Datenbank, die als Beispiel dienen
soll `kina'. Die Grundstruktur der Dokumente in dieser Datenbank ist folgendermaßen:

\begin{code}
{
   "_id": "fab911e6e9ed703bf536dccffe000937",
   "_rev": "1-183f162a09532ccacf8223f6e2ba95f6",
   "lang": "javascript"
}
\end{code}

Alle Beispiel-Requests werden unter Verwendung von cURL \cite{10} erstellt. Natürlich kann auch das von 
CouchDB mitgelieferte Webinterface `Futon' genutzt werden. Allerdings rate ich davon ab, da 
bei der Verwendung von cURL die Betrachtung der HTTP-Response zu wesentlich mehr Verständnis führt.
\end{abstract}

\section{Überblick}

CouchDB ist eine dokumentbasierte Datenbank und gehört zur Gruppe der NoSQL Datenbanken. CouchDB
ist ein Apache Open-Source Projekt und liegt aktuell in der Version 1.1.0 vor. Einige wichtige 
Merkmale von CouchDB sind:

\begin{itemize}
  \setlength{\parskip}{0pt}
  \item in Erlang \cite{4} geschrieben
  \item schemalos
  \item API: HTTP Rest \cite{5}
  \item implementiert ACID \cite{6} über MVCC \cite{7}
  \item CAP Theorem \cite{8}:
    \begin{itemize} 
      \item A (Availability) - ja
      \item P (Partition Tolerance) - ja
      \item C (Consistency) -> Eventual Consistency
    \end{itemize}
  \item MapReduce \cite{9} für Views
  \item Asynchrone Replikation in beide Richtungen - Master / Master
  \item Revisions Management
\end{itemize}

\section{Datenbanken}
\subsection{Alle Datenbanken im Cluster \htmlverb{GET} \htmlverb{\_all\_dbs}}

\begin{code}
curl -X GET http://127.0.0.1:5984/_all_dbs
["_replicator","_users"]
\end{code}

\subsection{Datenbank erstellen \htmlverb{PUT}}

\begin{code}
curl -X PUT http://127.0.0.1:5984/kina
{"ok": true}
\end{code}

\subsection{Datenbankinformationen erhalten \htmlverb{GET}}

\begin{code}
curl -X GET http://127.0.0.1:5984/kina
{"db_name":"kina",
 "doc_count":0,
 "doc_del_count":0,
 "update_seq":0,
 "purge_seq":0,
 "compact_running":false,
 "disk_size":79,
 "instance_start_time":"1312577294992829",
 "disk_format_version":5,
 "committed_update_seq":0}
\end{code}

\subsection{Datenbank löschen \htmlverb{DELETE}}

\begin{code}
curl -X DELETE http://127.0.0.1:5984/kina
{"ok": true}
\end{code}

\subsection{Datenbank replizieren \htmlverb{POST} \htmlverb{\_replicate}}
\emph{(options)} create\_target:true, cancel:true, continuous:true

\begin{code}
curl -X POST http://127.0.0.1:5984/_replicate 
     -H "content-type:application/json" 
     -d '{"source": "kina", 
          "target": "anik",
          "create_target": true}'
{"ok":true, "session_id":"f75eb944bac70f40e77953f484afb64c", 
 "source_last_seq":36, "history":
   [{"session_id":"f75eb944bac70f40e77953f484afb64c",
     "start_time":"Thu, 14 Apr 2011 20:36:12 GMT", 
     "end_time":"Thu, 14 Apr 2011 20:36:12 GMT",
     "start_last_seq":0,
     "end_last_seq":36,
     "recorded_seq":36,
     "missing_checked":0,
     "missing_found":14,
     "docs_read":14,
     "docs_written":14,
     "doc_write_failures":0
  }]
}
\end{code}

\subsection{Datenbank replizieren - filter}
\emph{(options)} filter:filter\_name, query\_params:\{filter\_name:query\_param\},\newline
doc\_ids: \brackets{docid\_1,docid\_2,docid\_n} \newline

Filter werden in \_design Dokumenten erstellt. Ein Filter kann später bei der Replikation genutzt werden, um nur
Dokumente zu replizieren, die den im Filter hinterlegten Kriterien entsprechen.

Filter erstellen:

\begin{code}
curl -X PUT http://127.0.0.1:5984/kina/_design/default \
  -H "content-type: application/json" \
  -d '{"filters":{
        "lang": 
          "function(doc, req) { 
            return \"javascript\" == doc.lang 
          }"
        }
     }'
\end{code}

Filter anwenden (davon ausgehend, es gibt drei Dokumente mit lang = javascript):

\begin{code}
curl -X POST http://127.0.0.1:5984/_replicate
     -H "content-type:application/json"
     -d '{"source": "kina",
          "target": "anik",
          "create_target": true,
          "filter": "default/lang"}'
{"ok":true,"session_id":"3546d87233636b04440e6e023c3e3018",
 "source_last_seq":6,
 "replication_id_version":2, "history":
  [{"session_id":"3546d87233636b04440e6e023c3e3018",
    [...]
    "docs_read":3,
    "docs_written":3,
    "doc_write_failures":0
  }]
}
\end{code}

Filter erstellen mit dynamischem Parameter (Auszug):

\begin{code}
curl [...]
  -d '{"filters":{
       "lang":·
         "function(doc, req) {·
           return req.query.lang == doc.lang·
         }"
       }
    }'
\end{code}

Filter anwenden (Auszug):

\begin{code}
curl [...]
  -d '{"source": "kina",
       "target": "anik",
       "create_target": true,
       "query_params": {
          "lang":"javascript"
       }
     }'
\end{code}

Filter andwenden unter Angabe von genauen Dokument IDs (Auszug):

\begin{code}
curl [...]
  -d '{"source": "kina",
       "target": "anik",
       "create_target": true,
       "doc_ids": ["id_1", "id_2", "id_n"]
     }'
\end{code}

\subsection{Änderungen in der DB verfolgen \htmlverb{GET} \htmlverb{\_changes}}
\emph{(params)} since=\brackets{sequence\_number}, style=all\_docs, limit=\brackets{n}, \newline
feed=\brackets{continuous,longpolling}, heartbeat=\brackets{milliseconds} \newline
filter=filter\_name, include\_docs=\brackets{true,false}, timeout=\brackets{milliseconds}

\begin{code}
curl -X GET "http://127.0.0.1:5984/kina/_changes \
  ?feed=continuous&heartbeat=2000"
{"seq":3,"id":"kn0001","changes": \
  [{"rev":"3-5570e8bbb34412db757c2df4bfa1099b"}]}
{"seq":4,"id":"fab911e6e9ed703bf536dccffe000937","changes": \ 
  [{"rev":"1-183f162a09532ccacf8223f6e2ba95f6"}]}
[...]
\end{code}

\subsection{Compaction: Datenbank \htmlverb{POST} \htmlverb{\_compact}}

\begin{code}
curl -X POST http://127.0.0.1:5984/kina/_compact \
  -H "content-type:application/json"
{"ok":true}
\end{code}

\subsection{Compaction: Design Dokumente \htmlverb{POST} \htmlverb{\_compact/design-doc}}

\begin{code}
curl -X POST http://127.0.0.1:5984/kina/_compact/default \
  -H "content-type:application/json"
{"ok":true}
\end{code}

\subsection{Compaction: Views \htmlverb{POST} \htmlverb{\_view\_cleanup}}

\begin{code}
curl -X POST http://127.0.0.1:5984/kina/_view_cleanup \
  -H "content-type:application/json"
{"ok":true}
\end{code}

\section{Dokumente}
Prinzipiell sollte beim Erstellen auf IDs zurückgegriffen werden, die CouchDB 
generiert. Allerdings macht es in manchen Fällen auch Sinn eigene IDs zu nutzen.

\subsection{UUIDs ausgeben \htmlverb{GET} \htmlverb{\_uuids}}
\emph{(params)} count={[}n{]}

\begin{code}
curl -X GET http://127.0.0.1:5984/_uuids?count=3
{"uuids":
  ["b7fc0ca7dffcc6d0f240c1e5bb000998",
   "b7fc0ca7dffcc6d0f240c1e5bb000fe8",
   "b7fc0ca7dffcc6d0f240c1e5bb001c9a"]
}
\end{code}

\subsection{Alle Dokumente erhalten \htmlverb{GET} \htmlverb{\_all\_docs}}
\emph{(params)} descending=true, key=\brackets{key}, startkey=\brackets{key}, startkey\_docid=\brackets{docid}, \newline
endkey=\brackets{key}, endkey\_docid=\brackets{docid}, \newline
group=\brackets{true,false}, group\_level=\brackets{0-n}, inclusive\_end=\brackets{true,false}, \newline
limit=\brackets{n}, reduce=\brackets{true,false}, skip=\brackets{n}, stale=ok, \newline
update\_seq=\brackets{true,false}

\begin{code}
curl -X GET http://127.0.0.1:5984/kina/_all_docs
{"total_rows":12,"offset":0,"rows":[
{"id":"7341477ce373f9cc76f351e5980008bb",
 "key":"7341477ce373f9cc76f351e5980008bb",
 "value":{"rev":"2-f0bfca3976ad04bce05b2ade242519d7"}},
{"id":"7341477ce373f9cc76f351e5980015cd",
 "key":"7341477ce373f9cc76f351e5980015cd",
 "value":{"rev":"2-d6f5f2cb326c1f68f95d2bfbef329280"}},
[...]
\end{code}

\subsection{Mehrere Dokumente mit key(id) erhalten \htmlverb{POST} \htmlverb{\_all\_docs}}
\emph{(params)} siehe: Alle Dokumente erhalten \newline
\emph{(!)} gut nutzbar, um mit einem Request mehrere Dokumente zu erhalten

\begin{code}
curl -X POST http://127.0.0.1:5984/kina/_all_docs?include_docs=true 
  -H "content-type:application/json" 
  -d '{"keys": [
        "fab911e6e9ed703bf536dccffe000937", 
        "7341477ce373f9cc76f351e5980015cd"
      ]}'
{"total_rows":12,"offset":0,"rows":[
  {"id":"fab911e6e9ed703bf536dccffe000937",
   "key":"fab911e6e9ed703bf536dccffe000937",
   "value":{"rev":"1-183f162a09532ccacf8223f6e2ba95f6"},
   "doc":{"_id":"fab911e6e9ed703bf536dccffe000937",
   "_rev":"1-183f162a09532ccacf8223f6e2ba95f6",
   "lang":"javascript"}},
  {"id":"7341477ce373f9cc76f351e5980015cd",
   "key":"7341477ce373f9cc76f351e5980015cd",
   "value":{"rev":"3-99b698d0caba3c9892a385dd0efe2a3d"},
   "doc":{"_id":"7341477ce373f9cc76f351e5980015cd",
   "_rev":"3-99b698d0caba3c9892a385dd0efe2a3d",
   "lang":"lisp"}}
]}
\end{code}

\subsection{Dokument erstellen mit CouchDB ID \htmlverb{PUT}}

\begin{code}
curl -X PUT http://127.0.0.1:5984/kina/ \
  b7fc0ca7dffcc6d0f240c1e5bb000998
{"ok":true,"id":"b7fc0ca7dffcc6d0f240c1e5bb000998",
 "rev":"1-967a00dff5e02add41819138abb3284d"}
\end{code}

\subsection{Dokument erstellen mit eigener ID \htmlverb{PUT}}

\begin{code}
curl -X PUT http://127.0.0.1:5984/kina/kn0001 -d '{}'
{"ok":true,"id":"kn0001",
 "rev":"3-5570e8bbb34412db757c2df4bfa1099b"}
\end{code}

\subsection{Dokument erstellen \htmlverb{POST}}

\begin{code}
curl -X POST http://127.0.0.1:5984/kina/
     -H "content-type: application/json"
     -d '{}'
{"ok":true,"id":"b7fc0ca7dffcc6d0f240c1e5bb000fe8",
 "rev":"1-367b00dfc5e02axd41819138abb3284d"}
\end{code}

\subsection{Dokument anfragen \htmlverb{GET}}
\emph{(params)} rev={[}revision{]}, revs=true, revs\_info=true

\begin{code}
curl -X GET http://127.0.0.1:5984/kina/ \
  b7fc0ca7dffcc6d0f240c1e5bb000998
{"_id":"b7fc0ca7dffcc6d0f240c1e5bb000998",
 "_rev":"1-367b00dfc5e02axd41819138abb3284d",
 "inhalt":"hier steht was"}
\end{code}

\subsection{Dokument erweitern / aktualisieren \htmlverb{PUT}}

\begin{code}
curl -X PUT http://127.0.0.1:5984/kina/ \
  b7fc0ca7dffcc6d0f240c1e5bb000998
     -d '{"_rev":"1-367b00dfc5e02axd41819138abb3284d",
          "neuer Inhalt": "kommt hier hin",
          "inhalt": "bestehender wird aktualisiert"}'
{"ok":true,"id":"7fc0ca7dffcc6d0f240c1e5bb000998",
 "rev":"2-53f21467344e4cb88384fc9e2e189049"}
\end{code}

\subsection{Dokument-Attachment speichern \htmlverb{PUT}}
\emph{(!)} generell könnte auch nur die Option \mono{data} für eine reine Textdatei verwendet werden

\begin{code}
curl -X PUT http://127.0.0.1:5984/kina/ \
  b7fc0ca7dffcc6d0f240c1e5bb000998/ \
  geburtstag.txt?rev=2-53f21467344e4cb88384fc9e2e189049
     --data-binary @geburtstag.txt
     -H "content-type: text/plain;charset=utf-8"
HTTP/1.1 201 Created
Server: CouchDB/1.1.0 (Erlang OTP/R14B03)
Location: http://127.0.0.1:5984/datenbankname/ \
  b7fc0ca7dffcc6d0f240c1e5bb000998/geburtstag.txt
Etag: "3-d5171a67e9e27aeba9beac32149a86a9"
Date: Mon, 07 Feb 2011 22:53:25 GMT
Content-Type: text/plain;charset=utf-8
Content-Length: 95
Cache-Control: must-revalidate

{"ok":true,"id":"b7fc0ca7dffcc6d0f240c1e5bb000998",
 "rev":"3-d5171a67e9e27aeba9beac32149a86a9"}
\end{code}

\subsection{Dokument löschen \htmlverb{DELETE}}
\emph{(!)} Dokument wird nur aus dem Index gelöscht und markiert mit \mono{deleted:true}

\begin{code}
curl -X DELETE http://127.0.0.1:5984/kina/ \
  7341477ce373f9cc76f351e598001cdd \
  ?rev=1-9533bf3b3c5cda717ed000b186fdafae
{"ok":true,"id":"7341477ce373f9cc76f351e598001cdd",
 "rev":"2-5c7fb5dfeaf6f7cea149922fa1cdaf96"}
\end{code}

\subsection{Dokument permanent löschen \htmlverb{POST} \htmlverb{\_purge}}
\emph{(!)} Refernez zum Dokument wird gelöscht; wird nicht repliziert, um den
Plattenplatz auch frei zu bekommen, muss \mono{\_compact} ausgeführt werden

\begin{code}
curl -X POST http://127.0.0.1:5984/kina/_purge/ 
  -H "content-type:application/json" 
  -d '{"7341477ce373f9cc76f351e598001cdd": 
        ["2-5c7fb5dfeaf6f7cea149922fa1cdaf96"]
     }'
{"purge_seq":1,"purged":{
  "7341477ce373f9cc76f351e598001cdd":
    ["2-5c7fb5dfeaf6f7cea149922fa1cdaf96"]
  }
}
\end{code}

\subsection{Dokument Info \htmlverb{HEAD}}

\subsection{Dokument kopieren \htmlverb{COPY}}

\subsection{Mehrere Dokumente anlegen \htmlverb{POST} \htmlverb{\_bulk\_docs}}
\emph{(options)} all\_or\_nothing:\brackets{true,false (=non-atomic=default)}, docs:\brackets{key:value,...}

\begin{code}
curl -X POST http://127.0.0.1:5984/kina/_bulk_docs \
  -H "content-type:application/json" \
  -d '{"docs":[
        {"lang":"c"}, 
        {"lang":"c++"}
     ]}'
[{"id":"7341477ce373f9cc76f351e5980008bb",
  "rev":"1-f491c132ea24ecf492e0b5ae18467876"},
 {"id":"7341477ce373f9cc76f351e5980015cd",
  "rev":"1-9533bf3b3c5cda717ed000b186fdafae"}]
\end{code}

\subsection{Mehrere Dokumente aktualisieren \htmlverb{POST} \htmlverb{\_bulk\_docs}}
\emph{(options)} all\_or\_nothing:\brackets{true,false (=non-atomic=default)}, docs:\brackets{key:value,...}

\begin{code}
curl -X POST http://127.0.0.1:5984/kina/_bulk_docs \
  -H "content-type:application/json" \
  -d '{"docs":[
        {"_id":"7341477ce373f9cc76f351e5980008bb",
         "_rev": "1-f491c132ea24ecf492e0b5ae18467876",
         "lang":"erlang"},
        {"_id":"7341477ce373f9cc76f351e5980015cd",
         "_rev":"1-9533bf3b3c5cda717ed000b186fdafae",
         "lang":"lisp"}
    ]}'
[{"id":"7341477ce373f9cc76f351e5980008bb",
  "rev":"2-f0bfca3976ad04bce05b2ade242519d7"},
 {"id":"7341477ce373f9cc76f351e5980015cd",
  "rev":"2-d6f5f2cb326c1f68f95d2bfbef329280"}]
\end{code}

\begin{thebibliography}{sotief}
\bibitem{1}\url{http://couchdb.apache.org}
\bibitem{2}\url{https://cloudant.com/futon/database.html?andywenk%2Fkina}
\bibitem{3}\url{http://www.couchbase.org/sites/default/files/uploads/all/documentation/couchbase-api.html}
\bibitem{4}\url{http://www.erlang.org}
\bibitem{5}\url{http://en.wikipedia.org/wiki/Representational_State_Transfer}
\bibitem{6}\url{http://en.wikipedia.org/wiki/ACID}
\bibitem{7}\url{http://en.wikipedia.org/wiki/Multiversion_concurrency_control}
\bibitem{8}\url{http://www.julianbrowne.com/article/viewer/brewers-cap-theorem}
\bibitem{9}\url{http://en.wikipedia.org/wiki/MapReduce}
\bibitem{10}\url{http://curl.haxx.se/}
\end{thebibliography}

\end{document}
